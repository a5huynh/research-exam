\documentclass[10pt,twocolumn]{article}

\usepackage{graphics}
\usepackage{color}

% Setup fonts to use
\usepackage{palatino}
%\usepackage[sfdefault]{quattrocento}

% Setup page geometry
\usepackage[margin=2cm]{geometry}

\begin{document}

\title{Crowdsourcing and Its Applications in Cultural Heritage}

\author{Andrew Huynh\\
University of California, San Diego\\
Department of Computer Science and Engineering\\
\texttt{a5huynh@cs.ucsd.edu}
}

\date{August 20, 2013}

\maketitle

\begin{abstract}
Massively parallel collaboration and emergent knowledge generation is described through a public survey for archaeological anomalies within ultra-high resolution earth-sensing satellite imagery.  Over 10K volunteers contributed 30K hours (3.4 years), examined 6,000 km$^2$, and generated 2.3 million feature categorizations. Motivated by the search for Genghis Khan's tomb, this effort seeks an enigma that lacks historical description of visual appearance. Consensus, defined by kernel density estimation, pools human perception for ``out of the ordinary" features across a vast landscape. The resulting map led a National Geographic expedition to confirm 55 archaeological sites.  A greater accuracy was observed in participants exposed to a self-evolving peer feedback loop, suggesting collective reasoning can emerge within networked groups to outperform the aggregate independent ability of individuals to define the unknown.
\end{abstract}

\section{Introduction}
\section{Incentivizing the Crowd}
\section{Learning from the Crowd}
\section{The Crowd \& The Machine}
\section{Applications in Cultural Heritage}
\section{Conclusion}
\section{Acknowledgements}

\end{document}

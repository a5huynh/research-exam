\documentclass[10pt,twocolumn]{article}

\usepackage{graphics}
\usepackage{color}

% Set it up so that cvites link to their respective citation in the
% bibliography. Also use sane color instead of green.
\usepackage[colorlinks,citecolor=blue]{hyper ref}

% Setup fonts to use
\usepackage{palatino}
%\usepackage[sfdefault]{quattrocento}

% Setup page geometry
\usepackage[margin=2cm]{geometry}

\begin{document}

\title{
	Crowdsourcing and Its Applications in Cultural Heritage}

\author{
	Andrew Huynh\\
	University of California, San Diego\\
	Department of Computer Science and Engineering\\
	\texttt{a5huynh@cs.ucsd.edu}}

\date{August 20, 2013}

\maketitle

\begin{abstract}
In the past decade, ``human computation'' or ``crowdsourcing'' has emerged as 
a new field of research leveraging the vast human connectivity offered by 
the Internet to solve problems that are often too large for individuals or 
too challenging for current automatic methods. While the physical preservation 
of cultural heritage has been an on-going process for decades, digital 
preservation techniques have only recently been applied to the protection, 
documentation, and interpretation of the world's cultural heritage. 
Crowdsourcing offers a unique way to involve the beneficiaries of the heritage 
being preserved in the process of the actual preservation. We present several 
examples of how crowdsourcing techniques is being utilized to protect, 
document, and understand cultural heritage. Finally, we discuss the 
implications of crowdsourcing on the future of cultural heritage preservation.
\end{abstract}

\section{Introduction}
The rise of "human computation" or "crowdsourcing" ~\cite{Howe2006} as a field
of research emerged from attempts to harness and motivate individuals to finish
small tasks by either creating enjoyable experiences~\cite{VonAhn2008}, 
appealing to scientific altruism~\cite{Cooper2010}, payment~\cite{Kittur2008},
or related services in exchange for participation~\cite{Hull2006}. 

These systems have been applied to tackle enormous problems ranging from image 
annotation~\cite{VonAhn2004}, galaxy classification~\cite{Lintott2008}, protein 
folding~\cite{Cooper2010}, and text transcription~\cite{VonAhn2008}. 
They have systematically demonstrated that reliable data can be collected in 
large amounts through incremental contributions from huge numbers of participants.

Growing in parallel with the rise in data collection 
capabilities~\cite{Crane2002}, digital cultural heritage refers to a variety of 
activities from the development of digital technologies used in exhibits to the 
collection~\cite{Proctor2010}, preservation~\cite{Chrons2011}, and 
interpretation of cultural heritage objects and locations from around the 
world. Specifically, the task of digitizing cultural heritage has had many 
successes with harnessing the power of "crowdsourcing." Indeed, the locality of 
cultural heritage and the impact it may have on the world presents a unique 
crowdsourcing task wherein the beneficiaries of the tasks may often be the 
participants themselves.

To understand how crowdsourcing has become the field it is today and its 
applications to cultural heritage, we start by examining the different methods 
of crowdsourcing and how they distribute small tasks that benefit from human 
involvement (Section~\ref{sec:active} \& \ref{sec:passive}). We can then 
examine how different approaches, ranging from simple clustering techniques to 
machine learning, can be applied to the crowdsourced data to extract insight 
(Section~\ref{sec:learning}). Following the discussion of those techniques, we 
examine how human involvement in those techniques through active feedback or 
even replacing components to the workflow can lead to even greater insight 
(Section~\ref{sec:the-crowd-machine}). Finally, we examine how the preservation 
of cultural heritage may benefit from the distributed human involvement of 
crowdsourcing (Section~\ref{sec:applications-heritage}).

\section{Active Crowdsourcing}
\label{sec:active}

Traditional ``active'' crowdsourcing techniques rely on a platform in which 
certain tasks can be requested for completion by the public~\cite{Howe2006}. 
This platform can be represented as a singular website dedicated to a specific 
task, such as galaxy classification in GalaxyZoo~\cite{Lintott2008}, or a 
larger market place in which many different types of tasks can be requested, 
such as the human intelligence tasks (HITs) in the Amazon Mechanical 
Turk~\footnote{http://www.mturk.com/mturk/} marketplace.

On a fundamental level, all these different platforms adhere to a set of
principles that enable individuals to finish a task in a meaningful manner and 
accomplish the overarching goal of the platform. A platform must utilize a 
method of engagement to attract participants, design tasks in such a way that 
participants can efficiently complete the task, and include constructs to 
ensure a high-quality of data output.

\subsection{Crowd Engagement}
\label{sec:engagement}

Engaging a large pool of labor and presenting compelling motivation to complete
is an essential piece of a platform's success. Here we examine three different
types of engagement: ``Value Incentives", ``The Greater Good", and ``Games With A 
Purpose."

\subsubsection*{Value Incentive}

Participants who can be engaged and motivated through \textit{Value Incentives} 
expect compensation that has a tangible value, which may come in one of two
forms:

\begin{itemize}
	\item
		\textbf{Cold Hard Cash} - Marketplaces such as Amazon Mechanical Turk
		present a financial reward in the exchange for completion of tasks.
	\item
		\textbf{Products and/or services} - The recent introduction of 
		``crowdfunding''~\cite{Belleflamme2011,Schwienbacher2010} platforms 
		such as Kickstarter~\footnote{http://www.kickstarter.com} allow the
		sourcing of money contributions instead of tasks in exchange for an 
		end product or service that is created utilizing the monetary
		contributions.
\end{itemize}

An impressive example of how powerful a value incentive may motivate people is
the 2009 Red Balloon Challenge. Teams competed for a \$40,000 prize to find 
10 weather balloons deployed at undisclosed locations across the U.S. A single
team found all balloons in the span of 8 hours using crowdsourcing with 
monetary incentivisation which divided the prize money with participants who
found the balloons~\cite{Tang2011}. This method quickly motivated thousands of 
participants to find the balloons. However, while value incentives may quickly 
motivate a large number of participants, it is important to note that the quality 
of the data received does not increase with the value offered~\cite{Mason2010a}.

\subsubsection*{The Greater Good}

In contrast to participants in \textit{Value Incentive} category, 
participants who are engaged and motivated to contribute to a platform through 
only the concept of pushing forward overarching goals without compensation 
can be seen as ``citizen scientists'', serving \textit{The Greater Good}. These 
citizen scientists often act as collaborators on these platforms volunteering 
their time to map streets around the world~\cite{Haklay}, classify 
galaxies~\cite{Lintott2008}, translate ancient 
texts~\footnote{http://www.ancientslives.org}, and even assess satellite 
imagery to aid emergency disaster response~\cite{Barrington2012,Goodchild2010}.

The earliest forms of citizen scientists can be traced back to the early 1900s 
with the Christmas Bird count run by the National Audubon 
Society~\cite{Silvertown2009}, where amateur birdwatchers contributed a major 
source of data used to determine the status of bird species in North 
America~\cite{Butcher1990}. More recently, the powerful connectivity of the 
internet has lead to larger scale citizen science projects contributing 
useful data to a number of fields. These collaborations between citizens 
and scientists have shown enormous potential~\cite{Newman2012} as a enhanced
method of collecting and interpreting scientific data.

\subsubsection*{Games With A Purpose}

Games With A Purpose (GWAPs)~\cite{VonAhn2008} attempts to engage individuals
through personal enjoyment or social reward by creating a game environmentt
in which crowdsourcing tasks may take place. GWAPs are often related to the 
process of ``gamification'', wherein game elements are added to what may be 
a mundane task, such as image annotation~\cite{VonAhn2004}, in order to motivate 
and engage participants in the task.


GWAPs can be and often are coupled with the previously mentioned \textit{Value 
Incentive} and \textit{The Greater Good} methodology as a means for additional 
motivation and engagement. Foldit~\cite{Cooper2010}, a crowdsourced effort to 
produce accurate protein structure models, successfully uses this model of
engagement attracting players who were not only interested in their potential
contributions to a scientific field but the interaction with other competitive
players in a game environment as well. In cultural heritage, \textit{Exploration:
Mongolia}~\footnote{http://exploration.nationalgeographic.com} uses this approach
to motivate ``armchair archaeologists'' to annotate satellite imagery to aid
the search the Tomb of Genghis Khan in a game environment.

\subsection{Quality Control}

A major issue often discussed in crowdsourcing research is the interesting 
data quality control problems that occur when accepting potentially 
inaccurate or even fraudulent data from human participants. Verifying the
quality of every submitted data point without existing automated methods can 
be as expensive as the process of collecting the data. As such, researchers
have used a variety of methods to ensure a higher quality of output from
participants.


\subsubsection*{Automatic Verification}
Crowdsourcing efforts like Foldit~\cite{Cooper2010}, while difficult to
formulate solutions involve problems with easily verifiable solutions that
can be automatically checked as soon as the task is completed.


\subsubsection*{Agreement}
Perhaps one of the first methods of quality control, examples of agreement
can be seen in the ESP game~\cite{VonAhn2004} and TagATune~\cite{Law2009a}.
Agreement accepts input if \textit{all} participants agree on the same 
solution for a task. 

Agreement can occur before or after a task, respectively
known as input agreement and output agreement. Input agreement, utilized in 
TagATune, involves giving independent participants inputs with that may or 
may not be the same and asked to describe the inputs to each other to 
determine the solution to the problem. Output agreement, utilized in the
ESP game, accepts a solution to a task only after two participants agree
on the same solution.


\subsubsection*{Redundancy}
Building upon agreement, redundancy as a form of quality control requires 
multiple participants to complete the same task~\cite{Sheng2008} selecting
a final solution through majority voting or averaging aggregate solutions.
Efforts such as reCAPTCHA~\cite{VonAhn2008} have successfully utilized this 
method to ensure high quality output, at the cost of additional crowdsourcing
work.


\subsubsection*{Reputation}
In reputation systems, a participants quality of work is determined through
the quality of previous inputs. Reputation can be used to immediately filter 
out participants who have a high potential of inaccurate input. This method 
is employed Amazon Mechanical Turk.


\subsubsection*{Expert/Crowdsourced Review}
Quality control using expert or crowdsourced review involve additional
sets of participants who review and check solutions for accuracy during 
or after the initial tasks are completed.

An example of crowdsourced review, Soylent~\cite{Bernstein2010} employs 
this method as the \textit{Find-Fix-Verify} pattern, splitting tasks 
into three stages that filter out inaccurate solutions at each stage 
through redundancy.


\subsubsection*{Statistical Filtering}
If the solution for a task is expected to match some distribution, 
such as in Quality of Experience evaluations~\cite{Chen2009}, researchers 
can filter out inaccurate data points.b


\subsubsection*{Annotation Models}
Modeling annotation processes~\cite{Karger,Welinder} using a set of variables 
provide a rich method of quality control through the representation of 
different factors that may lead to inaccurate data. Welinder et.\ 
al~\cite{Welinder} uses a model to discover and represent groups of 
participants with differing sets of skills and knowledge, increasing the 
quality of data that can be determined from each group. Karger et.\ 
al~\cite{Karger} takes this even further and uses a bipartite graph to model 
both tasks and participants to determine an optimal task assignment.


\section{Passive Crowdsourcing}
\label{sec:passive}

Crowdsourcing often invokes the image of large groups of participants in 
front of their computers independently working on a task. Passive 
crowdsourcing, ``work for nothing''~\cite{Adar2011}, relies on being able 
to gather data or perform computations that is produced as a result of 
the existing behavior of a group of individuals with or without their 
knowledge. Due to its passive, often hidden nature, passive crowdsourcing 
bypasses many of the problems due to crowd motivation and engagement 
present in active crowdsourcing techniques.


\subsection{Parasitic Computing~\cite{Barabasi2001}}
\label{sec:parastici-computing}
Parasitic computing leverages existing behavior of individuals to 
solve pieces of a complex computation problem, often without their
explicit permission. This has involved harnessing small amounts of
a participant's computational 
power~\cite{Anderson2002,Merelo2007,Merelo-Guervos2008} or by 
embedding the computations into an underlying standard 
protocol~\cite{Barabasi2001,Kohring2003}.

Initially, parasitic computing utilized underlying standard protocols
as a layer in which to do small pieces of computational work. For 
example, Barabasi~\cite{Barabasi2001} utilized the Hyper Text 
Transmission Protocol (HTTP) to perform calculations during the act 
of communication between web servers. Kohring~\cite{Kohring2003}
further explored this concept, embedding the computations as part of 
the standard IP communication protocol through the use of carefully
constructed ICMP packets.

More recent approaches harness idle or underutilized CPUs to do
pieces of computational work. Acting as a screensaver for idle 
computers SETI@home~\cite{Anderson2002} has been able to harness 
millions of computers around the world to analyze radio signals 
from space with the explicit permission of participants. Merelo 
et al.~\cite{Merelo2007,Merelo-Guervos2008} has explored this 
concept as a web application by running genetic algorithms in web 
browsers using a small percentage of computation resources as the 
result of web page being loaded.


\subsection{Crowdsensing}
\label{sec:crowdsensing}
In contrast to parasitic computing, crowdsensing~\cite{Ganti2011} 
involves the collection of data from participants through existing 
behavior as a method of providing a solution to those same participants.
This often involves, but is not limited to, embedded sensors on 
consumer-centric mobile devices or other computing devices with wireless 
capabilities that produce data as a participant interacts with the device 
or world through their existing behavior.

A remarkable example of crowdsensing without mobile sensors is
Google Flu Trends (GFT)~\footnote{http://www.google.org/flutrends}. 
GFT uses search query data for influenza related search terms provided 
by millions of Google users to estimate influenza activity. These 
results have been shown as an accurate predictor of influenza 
trends~\cite{Dugas2012} accurately tracking influenza even before 
national disease centers.

Building upon the existing need and activity of checking
traffic, Google Traffic~\cite{GoogleTraffic} and VTrack~\cite{Thiagarajan2009}
crowd sense real-time traffic information through the act of
using their map or traffic checking application.

A non-intuitive, yet powerful example of crowdsensing relates
to 3D reconstruction from photographs~\cite{Agarwal2009,Frahm2010},
which requires large amounts of imagery, has benefitted from
crowdsensing through the use of publicly available photographs
on websites such as Flickr~\footnote{http://www.flickr.com}. This 
example is especially useful in digital preservation, where 
publicly available photos, such as those provided by tourists,
of important cultural heritage sites could be used to construct
models for preservation purposes.


\section{Learning from the Crowd}
\label{sec:learning}

Crowdsourcing has the potential to collect an enormous amount of
information, all of which can be harnessed to partially automate
the tasks from which the data originated through the use of 
machine learning. Machine learning approaches excel at recognizing
consistent patterns among labeled training examples, such as those
provided by crowdsourcing participants, which can then be used to 
label new, unlabeled data.

Pattern recognition techniques rely on the extraction of features
that describe the properties of the concept being recognized. For
example, feature extraction in images can refer to edges or corners 
seen in the image or color histograms. Machine learning algorithms
can then identify statistical consistencies between sets of
features and can apply this knowledge to future sets of new,
unlabeled data.

While traditional machine learning approaches focus on a single,
fixed training set to learn static models, crowdsourced data presents
a source of temporally dynamic data which can be used to continually
create new, improved models. This methodology, known as active 
learning~\cite{Settles2010}, has been shown to produce reliable,
accurate models~\cite{Barrington2012}.

\subsection{Active Learning}

The key idea behind active learning is that a machine learning algorithm can
achieve greater accuracy with fewer training labels if it is allowed to choose the
data from which it learns.

\section{Applications in Cultural Heritage}
\label{sec:applications-heritage}

\section{Conclusion}
\section{Acknowledgements}
Thanks to Professor Yoav Freund for chairing my research exam committee and 
for Serge Belongie and Bill Griswold for participating on the committee.
And most of all, thanks to the MTurkers who find research papers for certain 
sections of this exam.

\bibliography{bibs/ActiveLearning,bibs/Andrew-HCOMP,bibs/Andrew-CogSci,bibs/CulturalHeritage,bibs/hcomp,bibs/Web}
\bibliographystyle{plain}

\end{document}
